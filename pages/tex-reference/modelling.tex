\documentclass{article}
\usepackage{graphicx, geometry,amsmath}
\geometry{margin = 2cm}
\begin{document}
    Note: \( (r,\theta , \phi) \) centred on \( (0,0,f) \)

    For an incident beam with intensity distribution \(I_0(\theta ) \) along \(\hat{z}\) axis, a paraboloid reflector: \(z^2 = K(x^2+y^2)\) which has focal point at (0,0,f):
    
    Near focus:
    \[ \bar{E}(r_0,\theta_0 ,\phi_0 ) = \frac{1}{2\pi i} \int_0^{2 \pi } \int_0^{\theta_{max}} \hat{a} (\theta , \phi) I_0(\theta) q(\theta) e^{-i K \bar{s}_0 \cdot \bar{r}_0} sin(\theta) \; {d\theta} \; {d\phi} \]
    \begin{itemize}
	\item \( \hat{a}   (\theta , \phi) \) is the polarization direction
	\item \( \hat{s} (\theta , \phi) \) is the unit vector in the direction of propagation
	\item \( \bar{r}_0 \) is the position vector of point of measurement (point \( (r_0, \theta_0 , \phi_0)\))
	\item \( q(\theta) \) is the apodization factor of a paraboloid
    \end{itemize}

    Note: \(P\) depends on \(\theta\) as \( P(\theta) = f \, tan \theta - K \, P(\theta)^2 \, tan \theta = 2f \, \left(\frac{sec \theta - 1}{tan \theta}\right) = 2f \, tan\left(\frac{\theta}{2}\right)\)

    The focal sphere is a sphere centred on \( (0,0,f) \) with radius \(f\). Consider an infinitesimally thin annulus of radius \(R\) taken from the incident beam. All of this light is focused into a strip on the focal sphere. The apodization factor, \(q(\theta)\), ensures conservation of energy since it is defined as the ratio \(\sqrt{\frac{dS_i}{dS_s}}\) where \(dS_i\) is an infinitesimal area of the incident beam (annulus) and \(dS_s\) is the corresponding area on the focal sphere.
    \[dS_i = 2 \pi R \; dR\]
    \[ R = P(\theta) = 2f \, tan\left(\frac{\theta}{2}\right)\]
    \[ dR = f \, sec^2\left(\frac{\theta}{2}\right) \; d\theta \]
    \[ dS_i = 4 \pi f^2 \, tan\left(\frac{\theta}{2}\right) \, sec^2\left(\frac{\theta}{2}\right) \; d\theta \]
    \[ dS_s = (2 \pi f) (sin \theta) (f \, d\theta) = 2 \pi f^2 \, sin \theta  \; d\theta \]
    \[ q(\theta) = \sqrt{\frac{dS_i}{dS_s}} = \sqrt{\frac{4 \pi f^2 \, tan\left(\frac{\theta}{2}\right) sec^2\left(\frac{\theta}{2}\right) \; d\theta}{2 \pi f^2 \, sin \theta \; d\theta}} = \sqrt{\frac{2 tan\left(\frac{\theta}{2}\right) sec^2\left(\frac{\theta}{2}\right) }{2 sin\left(\frac{\theta}{2}\right) cos\left(\frac{\theta}{2}\right)}} = sec^2\left(\frac{\theta}{2}\right) \]


    By convention: \[ \bar{r}_0 = r_0 \begin{pmatrix} sin\theta_0 \, cos\phi_0 \\ sin\theta_0 \, sin\phi_0 \\ cos\theta_0\end{pmatrix}\]

    For incident rays parallel to \(\hat{z}\): \[\bar{s} = \begin{pmatrix} sin\theta \, cos\phi \\ sin\theta \, sin\phi \\ cos\theta\end{pmatrix} \]

    \begin{align*}
    e^{-iK\bar{s}\cdot\bar{r}_0} & = e^{-i K r_0 (sin\theta \, sin\theta_0 \, [cos\phi\, cos\phi_0\, + sin\phi\, sin\phi_0] + cos\theta\, cos\theta_0)}\\ & = e^{-iK(sin\theta\, sin\theta_0\,cos(\phi-\phi_0)+cos\theta\,cos\theta_0)}
    \end{align*}
    \begin{itemize}
        \item Before reflection, \(\hat{a}_0 = \begin{pmatrix} cos\phi \\ sin\phi \\ 0 \end{pmatrix}\)
        \item After reflection, \(\hat{a}(\theta,\phi) = \begin{pmatrix} cos\theta\, cos\phi \\ cos\theta\, sin\phi \\ sin\theta \end{pmatrix}\)
    \end{itemize}
    
    Therefore:
    \[\bar{E}(r_0, \theta_0,\phi_0) = \frac{Kf}{2\pi i}\int_0^{2\pi} \int_0^{\theta_{max}}I_0(\theta)sec^2\left(\frac{\theta}{2}\right)sin\theta e^{-iKr_0[sin\theta sin\theta_0 cos(\phi - \phi_0) + cos\theta cos\theta_0]} \hat{a}(\theta,\phi) \; d\theta \; d\phi\]
    
    Note the following identities for the n\textsuperscript{th} bessel function of the first kind: \(J_n(x)\)
    \[\int_0^{2\pi} cos(n\varphi)e^{ixcos(\varphi-\varepsilon)}d\varphi = 2\pi i^n J_n(x) cos(n\varphi)\]
    \[\int_0^{2\pi} sin(n\varphi)e^{ixcos(\varphi-\varepsilon)}d\varphi = 2\pi i^n J_n(x) sin(n\varphi)\]
    
    Thus:
    \begin{align*}
        \bar{E}(r_0,\theta_0,\phi_0) &= \frac{Kf}{2\pi i} \int_0^{\theta_{max}} I_0(\theta) sin\theta \, sec^2\left(\frac{\theta}{2}\right)e^{-iKr_0 cos\theta_0 cos\theta} cos\theta \int_0^{2\pi} \begin{pmatrix} cos\phi \\ sin\phi \\ tan\theta \end{pmatrix} e^{-iKr_0(sin\theta_0sin\theta cos(\phi-\phi_0))}d\phi \; d\theta \\ &= Kf \int_0^{\theta_{max}} I_0(\theta) sin\theta \, sec^2\left(\frac{\theta}{2}\right)e^{-iKr_0 cos\theta_0 cos\theta} \begin{pmatrix} J_1(Kr_0sin\theta_0sin\theta)cos\phi_0 \\ J_1(Kr_0sin\theta_0sin\theta)sin\phi_0 \\ -i J_0(Kr_0sin\theta_0sin\theta)tan\theta \end{pmatrix} d\theta
    \end{align*}
    
    Represented in cylindrical coordinates:
    \[E_z(r_0,\theta_0,\phi_0) = \bar{E} \cdot \hat{z} = -iKf \int_0^{\theta_{max}} I_0(\theta)sec^2\left(\frac{\theta}{2}\right) sin^2\theta e^{-iKr_0 cos\theta_0 cos\theta} J_0(Kr_0sin\theta_0sin\theta) \; d\theta\]
    \[E_r(r_0,\theta_0,\phi_0) = \bar{E} \cdot \hat{r} = \bar{E} \cdot \begin{pmatrix} cos\phi_0 \\ sin\phi_0 \\ 0 \end{pmatrix}\]
    \[= \frac{Kf}{2} \int_0^{\theta_{max}} I_0(\theta) sec^2\left(\frac{\theta}{2}\right) sin(2\theta) e^{-iKr_0 cos\theta_0 cos\theta} J_1(Kr_0sin\theta_0 sin\theta) \; d\theta\]
    \[E_{\phi}(r_0,\theta_0,\phi_0) = \bar{E} \cdot \hat{\phi} = \bar{E} \cdot \begin{pmatrix} -sin\phi_0 \\ cos\phi_0 \\ 0 \end{pmatrix} = 0 \]
    
    For an angle of incidence \(\alpha \neq 0\):

    Change of coordinates:
    \[r_0 \mapsto r_0'(r_0,\theta,\phi_0)\]
    \[\phi_0 \mapsto \phi_0'(r_0,\theta,\phi_0)\]
    
    Such that the integrals for \(\bar{E}\) take the same form.
    Additional phase factor: \(e^{-iK\zeta(\theta,\phi,\alpha)} = e^{-iK(2f\,tan\left(\frac{\theta}{2}\right)\, \alpha \, cos\phi)}\)

    Therefore we require an aberration function, \(\zeta(\theta,\phi,\alpha)\). Thus the exponential term is now:
    \[e^{-iK(sin\theta(f\alpha \frac{tan\left(\frac{\theta}{2}\right)cos\phi}{sin\theta} - r_0 cos\phi_0 cos_\phi - r_0 sin\phi_0 sin\phi)sin\theta_0 + r_0 cos\theta_0 cos\theta)}\]
    \[= e^{-iK[r_0'(sin\theta sin\theta_0 cos(\phi - \phi_0'))+r_0 cos\theta_0 cos\theta]}\]
    
    Recall \(a\,cos\phi + b\,sin\phi = \sqrt{a^2 + b^2} \; cos\left(\phi - tan^{-1}\left(\frac{b}{a}\right)\right)\)

    Substituting in values for \(a\) and \(b\):
    \[a = \frac{f\,\alpha\,tan\left(\frac{\theta}{2}\right)}{sin\theta} - r_0 cos\phi_0\]
    \[b = -r_0 sin\phi_0\]
    \[r_0' = \sqrt{a^2 + b^2} = \sqrt{\frac{f^2 \alpha^2 tan^2}{sin^2(\theta)}+r_0^2 - 2f\alpha r_0 cos\phi_0 \frac{tan\left(\frac{\theta}{2}\right)}{sin\theta}}\]
    \[tan(\phi_0') = \frac{r_0 sin\phi_0}{r_0 cos\phi_0 - \frac{f \alpha tan\left(\frac{\theta}{2}\right)}{sin\theta}}\]
    \[sin(\phi_0') = \frac{r_0 sin\phi_0}{r_0'}\]
    \[cos(\phi_0') = \frac{(r_0 cos\phi_0 - \frac{f \alpha tan\left(\frac{\theta}{2}\right)}{sin\theta})}{r_0'}\]

    As before, using the new coordinates:
    \[\bar{E} = Kf \int_0^{\theta_{max}} I_0(\theta) sin\theta \, sec^2\left(\frac{\theta}{2}\right) cos\theta e^{-iK r_0' cos\theta_0 cos\theta} \begin{pmatrix} J_1(Kr_0' sin\theta_0 sin\theta)cos\phi_0' \\ J_1(Kr_0' sin\theta_0sin\theta)sin\phi_0' \\ -i J_0(Kr_0' sin\theta_0sin\theta)tan\theta \end{pmatrix} d\theta \]

    We take the intensity distribution, \(I_0(\theta)\), to be:
    \[I_0(\theta) = Ne^{-\left(\frac{r(\theta)}{w_0}\right)^2} J_1\left(\frac{2r(\theta)}{w_0}\right)\]
\end{document}